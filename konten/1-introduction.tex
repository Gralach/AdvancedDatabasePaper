% Ubah judul dan label berikut sesuai dengan yang diinginkan.
\section{Introduction}
\label{sec:introduction}

% Ubah paragraf-paragraf pada bagian ini sesuai dengan yang diinginkan.
\par SQL injection (SQLi) is an attack technique that allows malicious actors to inject code on a database server through a web application's input field \cite{7395166}. This technique can be destructive to a database, making malicious actors capable of gaining access to sensitive information, destroying data, or understanding the whole system.
\par The prevention of SQLi attacks is critical for web developers and security professionals. According to OWASP 2021 \cite{OWASP}, SQLi is listed as the top 3 out of 10 most serious web application security risks with a max incident rate of 19\% and an average incidence rate of 3.37\%. The common practice of SQLi attacks prevention is to use input validation and sanitization, which directly detect and filter out malicious input before it gets processed by the database server. But even these techniques still have limitations in handling complex SQLi and need more effort to update the techniques to keep handling the SQLi since malicious actors always find ways to break the security.

\par In this modern era, machine learning exists to be a promising alternative for detecting and preventing SQLi attacks. SQLi attacks from the past can be used as a dataset for machine learning to learn and become capable of preventing SQLi attacks in the future. With this approach, machine learning can adapt to new types of SQLi attacks by learning and evolving, making malicious actors have difficulty bypassing the database server. Furthermore, this approach may handle more complex and shady input from malicious actors rather than only using the common practice.

\par In this report, we will implement two-step SQLi prevention, which combines keyword filtering with blacklist and machine learning prediction with LSTM. Besides evaluating the model's predictive performance, we will also review the prediction time by comparing it with various classical machine learning algorithms.

\par This report is organised as follows. Section \ref{sec:related-works} will discuss three related state-of-the-art approaches in this area. Section \ref{sec:prerequisites} will provide the key concepts related to this project. Section \ref{sec:proposed-design} will describe our overall approach and delve more into our proposed two-step prevention. In Section \ref{sec:experiment-setup}, we will describe the experimental setup for our approach. In Section \ref{sec:result}, we will present our experiments' results and compare them to the state-of-the-art approaches listed in Section II. Lastly, in Section \ref{sec:conclusion}, we will conclude this paper.
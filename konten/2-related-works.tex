% Ubah judul dan label berikut sesuai dengan yang diinginkan.
\section{Related Works}
\label{sec:related-works}

% Ubah paragraf-paragraf pada bagian ini sesuai dengan yang diinginkan.

\par Many researchers have researched the method of detecting SQLi attacks using machine learning, with each proposing new methods and applying different algorithms. With many types of research already published with the same goal, we will review their approach and methodology in this section.
\par Li et al. \cite{8854182} proposed a deep forest-based SQLi method framework. Their framework consists of two stages: offline training and online testing. Both stages are similar in that they involve preprocessing, feature extraction, and generating vectors. The offline stage includes a data collection step, while the online stage must output a result (malicious or non-malicious) using the trained model. The Adaptive Deep Forest Model (ADF) is the main component of their algorithm. Compared to other algorithms, this has the advantage of fully utilizing the information because the model's depth is sufficiently complex, and vectors are processed by layers while being able to complete intra-layer transformation. The ADF also allows the model to ignore parts of the feature vectors that may affect the model. Li et al. also introduced the AdaBoost algorithm, which improves the ability of the deep forest model to self-adjust. Within each level of the deep forest, the weights of the features are updated based on the classification error rate. This is particularly useful in solving the imbalanced distribution of multi-type samples. The ADF allows the model parameters to be adjusted automatically in the training process, which improves detection accuracy.
\par Hasan, Balbahaith, and Tarique \cite{8959617}, similar to ours and the previous work's approach, used a feature extractor where the extracted features are used to determine whether a query is malicious or not. Their proposed system is placed between the application and the system to evaluate incoming queries before forwarding them to the database. To decide which algorithms are best suited for their system, they evaluated the performance of different classifiers before finally choosing the Ensemble Boosted and Bagged Trees classifiers because they had the highest accuracy. The researchers of this paper also developed a graphical user interface (GUI) for users to enter queries and have them classified by their system. Although this paper compares the performance of 23 algorithms, the dataset size is small (616 statements), and the final accuracy is also taken from the training data, so it does not show a good representation of the actual result.
\par Li et al. \cite{8616823} discuss a method called long short-term memory (LSTM) that has been applied in our paper and is heavily used in deep learning. Like previous works, LSTM is used as a feature extractor and trains the classifiers for SQLi attacks. In this paper, the feature vector used is called word2vec because it yielded better results than the bag-of-words approach. In our paper, we implemented the term frequency-inverse document frequency (TF-IDF). This paper also made use of six datasets, with the first one being used for training. From the data split of all the datasets, Dataset 3 has a similar data structure to ours. Therefore, we can compare the results between our proposed method and theirs.
\par With many different approaches for SQLi detection, most have a similar approach where it starts with string matching or feature extraction. However, with the increase in technology literacy, meaning that hackers are becoming more intelligent, security must also become more advanced. In this paper, we propose a two-step approach by incorporating keyword filtering and model prediction within one architecture.

